% Template Tesi in Fisica (versione completa in un unico file)
%==============================================================
\documentclass[12pt]{report}

% Margini
\usepackage[top=2.5cm, bottom=2.5cm, left=4cm, right=2.5cm, centering, head=21.75 pt]{geometry}

% Interlinea
\linespread{1.5}

% Librerie utili
\usepackage{amsmath}
\usepackage{amsthm}
\usepackage{subcaption}
\usepackage{float}
\usepackage{caption}
\usepackage{subcaption}
\usepackage{booktabs}
\usepackage{siunitx}
\usepackage{caption}
\usepackage{amssymb}
\usepackage[italian]{babel}
\usepackage[utf8]{inputenc}
\usepackage{scrlayer-scrpage}
\ifoot[]{}
\cfoot[]{}
\ofoot[\pagemark]{\pagemark}
\pagestyle{scrplain}
\usepackage{mathptmx}
\usepackage[hidelinks]{hyperref}
\usepackage{graphicx}
\usepackage{csquotes}
\usepackage[backend=biber, sorting=nty]{biblatex}
\appto{\bibsetup}{\raggedright}
\addbibresource{bibliography.bib}

\usepackage{titlesec}
\usepackage{float}
\usepackage{listings}
\usepackage{xcolor}

% Stile del codice
\definecolor{mygreen}{rgb}{0,0.6,0}
\definecolor{mygray}{rgb}{0.5,0.5,0.5}
\definecolor{mymauve}{rgb}{0.58,0,0.82}
\definecolor{darkgray}{rgb}{.4,.4,.4}
\definecolor{navy}{HTML}{000080}
\definecolor{purple}{rgb}{0.65, 0.12, 0.82}
\definecolor{codepurple}{rgb}{0.58,0,0.82}
\definecolor{backcolour}{rgb}{0.95,0.95,0.92}

% Formato intestazioni capitoli
\titleformat{\chapter}[block]{\normalfont\LARGE\bfseries}{\thechapter.}{0.5em}{\LARGE}
\titlespacing*{\chapter}{0pt}{-20pt}{25pt}

\begin{document}

%---------------------------------------------------------------
% FRONTESPIZIO
%---------------------------------------------------------------
\begin{titlepage}
\begin{center}
    {\LARGE{Università degli Studi di Milano-Bicocca \\}}
    {\small{Dipartimento di Fisica "Giuseppe Occhialini"}}\\
    {\small{Corso di Laurea Triennale}}
\end{center}

\begin{figure}[H]
\centering
\includegraphics[width=0.4\textwidth]{logo.png}
\end{figure}

\begin{center}
    {\Large {Analisi spettrale di eventi SEP e stime della dose per missioni interplanetarie}}
\end{center}

\vspace{2cm}

\begin{minipage}[t]{0.47\textwidth}
{\large{\bf Relatore: \\ Stefano Della Torre}}\\[0.5cm]
{\large{\bf Correlatore: \\ Massimo Gervasi}}
\end{minipage}
\hfill
\begin{minipage}[t]{0.47\textwidth}
\raggedleft
{\large{\bf Candidato: \\ Stefano Dolci}}
\end{minipage}

\vspace{25mm}
\centering{\large{\bf ANNO ACCADEMICO 2024/2025}}
\end{titlepage}

\tableofcontents
\clearpage

\clearpage
\setcounter{page}{1}

%---------------------------------------------------------------
% CAPITOLO 1 — INTRODUZIONE
%---------------------------------------------------------------
\chapter{Introduzione}
Nello spazio interplanetario sono principalmente due le sorgenti di radiazione ionizzante per le sonde spaziali: i raggi cosmici galattici (GCR) e le particelle energetiche solari (SEP). Quest'ultime sono particelle ad alta energia (da KeV fino a GeV), emesse durante eventi di attività solare, principalmente protoni (90\% del flusso) ma anche ioni pesanti ed elettroni.
Le SEP sono note per avere energie inferiori rispetto ai GCR, ma la natura imprevedibile delle attività solari a loro legate le rende un rischio significativo per le missioni spaziali, sopratutto quelle interplanetarie dove la magnetosfera terrestre non può offrire protezione. Lo scopo di questo elaborato è proprio quello di analizzare in dettaglio alcuni eventi SEP noti nella storia recente e di stimare la dose (TID) che una sonda interplanetaria riceverebbe durante una tipica missione verso Marte.








\section{Particelle Energetiche Solari (SEP)}

La fisica delle SEP è uno dei campi dell'eliofisica, un nuovo termine definito dalla Nasa all'inizio di questo millenio \cite{nasa_ref} per indicare la scienza del Sole congiunta con i suoi effetti sul'ambiente interplanetario. È una disciplina relativamente giovane, agli inizi degli anni 50' Scott Forbush riportò per la prima volta un evento SEP come un improvviso aumento del flusso di particelle energetiche che raggiungevano le camere a nebbia a terra. In quegli anni si conoscevano come candidati solamente i brillamenti solari, noti come "Flare", ovvero impulsive esplosioni visibili di energia in prossimità della corona solare.
Con l'avvento dei satelliti per osservazioni solari, ad oggi sappiamo che esistono due categorie di SEP:  
\begin{itemize}

    \item[-] SEP impulsivi: brevi e intensi aumenti del flusso di particelle, il plasma intrappolato nelle linee di campo chiuse ad arco del Sole viene accellerato verso la superfice e espulso, questi eventi sono legati prorio ai flare.
    

    \item[-] SEP graduali: si tratta di eventi più lunghi che si possono estendere per più giorni, sono legati alle espulsioni di massa coronale (CME), quando le linee di campo vengono stirate dal vento solare possono rompersi e il plasma intrappolato viene espluso, le particelle presenti nel tragitto vengono accellerate dall'onda d'urto e guidate verso lo spazio interplanetario. 
\end{itemize}


\begin{figure}[H]
\centering
\includegraphics[width=0.8\textwidth]{figures/Tipologia_SEP.png}
\caption{Tipologia di eventi SEP, a sinistra evento impulsivo (Flare) a destra evento graduale (CME). Immagine adattata da "Solar Energetic Particles - Reames" \cite{Solar_Energetic_Particles_Reames}.}
\label{fig:tipologia_sep}
\end{figure}




\section{Danno da radiazione sull'elettronica}


\section{Missione interplanetaria verso Marte}

\section{Obiettivi della tesi}

%---------------------------------------------------------------
% CAPITOLO 2 — CATALOGO 6 EVENTI SEP
%---------------------------------------------------------------
\chapter{Dati e analisi spettrale}
Descrizione dei 6 eventi SEP scelti per analisi dettagliata


\section{Dati e strumentazione}


\section{Selezione 8 eventi SEP}

\begin{itemize}
    \item[-] AUG 1972
    \item[-] OCT 1989 
    \item[-] OCT 2003
    \item[-] JAN 2005
    \item[-] JUL 2012
    \item[-] SEP 2017
    \item[-] OCT 2021
    \item[-] MAY 2024
\end{itemize}


\section{Fit spettrale}

Satelliti, strumentazione e detector utilizzati
Range energetico dei dati 
Metodi di pulizia e filtraggio dati 
Durata evento
Incertezze
Calcolo fluenza e FIT con modelli 
Eventuali check e confronti letteratura

%---------------------------------------------------------------
% CAPITOLO 3 — CALCOLO DOSE SRNIEL
%---------------------------------------------------------------
\chapter{Calcolo della dose SRNIEL}

\section{Modello simulazione SRNIEL}

\section{Geometrie e modello della sonda}


\section{Propagazione delle incertezze}

\section{Risultati dosi stimate}
Calcoli dose e propagazione incertezze 
Modello shielding planare e sferico utilizzati per sonda


%---------------------------------------------------------------
% CAPITOLO 4 — ANALISI PROBABILISTICA}
%---------------------------------------------------------------
\chapter{Funzione di trasferimento}




\section{Funzione dose-fluenza D(F)}


\section{Influenza della metrica}




 

%---------------------------------------------------------------
% CAPITOLO 5 — CONFRONTO RISULTATI CON MISSIONI REALI
%---------------------------------------------------------------
\chapter{Stime della dose per la missione}

\section{Modelli probabilistici SEP}


\section{Parametri missionistici}


\section{Simulazioni SAPPHIRE}



\section{Validazione con missioni reali}


\cite{zeitlin2013}

%---------------------------------------------------------------
% CAPITOLO 6 — CONCLUSIONI
%---------------------------------------------------------------
\chapter{Conclusioni e miglioramenti}

Sintesi dei risultati ottenuti
confronto SEP vs GCR
Punti critici di queste stime 
Possibili miglioramenti generali per il lavoro di Tesi 

%---------------------------------------------------------------
% BIBLIOGRAFIA
%---------------------------------------------------------------
\printbibliography[heading=bibintoc, title={Bibliografia}]

%---------------------------------------------------------------
% APPENDICE
%---------------------------------------------------------------
\appendix
\chapter{Appendice}
Materiale aggiuntivo, formule, grafici supplementari.

\end{document}