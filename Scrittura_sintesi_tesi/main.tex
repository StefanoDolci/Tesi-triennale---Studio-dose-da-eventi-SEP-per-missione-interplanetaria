\documentclass[a4paper,12pt]{article}

\usepackage[utf8]{inputenc}
\usepackage[T1]{fontenc}
\usepackage[italian]{babel}
\usepackage{geometry}
\geometry{margin=2.5cm}
\usepackage{setspace}
\onehalfspacing

\usepackage{amsmath,amssymb}
\usepackage{graphicx}
\usepackage{hyperref}

\begin{document}

\begin{center}
    {\Large \textbf{Sintesi della Prova Finale}}\\[1.5cm]

    \begin{tabular}{ll}
        \textbf{Cognome e Nome:}   & Dolci Stefano \\[4pt]
        \textbf{Matricola:}        & 900231 \\[4pt]
        \textbf{Corso di Laurea:}  & Fisica \\[4pt]
        \textbf{Titolo:}           & Analisi spettrale di eventi SEP e stime \\
                                   & della dose per missioni interplanetarie \\[4pt]
        \textbf{Relatore:}         & Prof. Gervasi Massimo \\[4pt]
        \textbf{Correlatore:}      & Prof. Della Torre Stefano \\[4pt]
        \textbf{Data della seduta:} & 25 Marzo 2026 \\[4pt]
        \textbf{Recapito telefonico:} & +39 3343715962 \\[4pt]
    \end{tabular}
\end{center}

\vspace{0.5cm}

\section*{Introduzione}
Nello spazio interplanetario, la radiazione cosmica è uno dei principali limiti per il successo delle missioni spaziali. Al di fuori della magnetosfera terrestre sono due le principali sorgenti di radiazione: i raggi cosmici galattici (GCR) e le particelle energetiche solari (SEP). I GCR sono particelle ad alta energia originate al di fuori del sistema solare, il loro flusso è isotropico e relativamente costante, per questo motivo rimangono uno dei principali rischi per la progettazione di missioni a lungo termine. D'altro canto i SEP sono particelle energetiche, tipicamente protoni, elettroni e ioni pesanti, che vengono accelerate come risultato di improvvise ed energetiche eruzioni solari. Nonostante la maggior parte di questi eventi pone un rischio inferiore rispetto a quello dei GCR, durante i periodi di massima attività solare alcuni tra questi possono raggiungere flussi estremi in poche ore, diventando la componente dominante nel budget di radiazione per le missioni. La natura imprevedibile dei SEP richiede inevitabilmente un approccio probabilistico quando ci si pone l'obiettivo di stimare il rischio da tali eventi. Nel seguente elaborato si è cercato di fare proprio questo, considerando solo il contributo da SEP stimare la dose che un'ipotetica sonda interplanetaria riceverebbe durante una tipica missione verso Marte. Per farlo sono stati analizzati gli spettri in energia da 8 eventi SEP, i più intensi e noti degli ultimi 5 cicli solari (SC20-SC25). Dagli spettri, utilizzando il calcolatore SR-NIEL, è stata stimata la \textit{Total Ionizing Dose (TID)} in silicio dietro a schermature di alluminio in varie geometrie. Infine combinando questi dati sperimentali con il modello probabilistico SAPPHIRE sono state stimate le dosi entro un certo intervallo di confidenza per vari scenari di missione.

\section*{Risultato finale}
In questa sezione vengono riportati i risultati più significativi di questo elaborato. È stato stimato che durante un periodo del massimo solare, per un esposizione di 0.5 anni durante il transito verso Marte, la TID in silicio dietro a una schermatura sferica di 10 g/cm$^2$ è pari a $43.4\pm4.3$ mGy con un livello di confidenza al 90\%.


\section*{Conseguenze}



\section*{Metodologia}



\end{document}
